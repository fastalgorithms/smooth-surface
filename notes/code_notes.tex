\documentclass[11pt, oneside]{article}   	% use "amsart" instead of "article" for AMSLaTeX format
\usepackage{geometry}                		% See geometry.pdf to learn the layout options. There are lots.
\geometry{letterpaper}                   		% ... or a4paper or a5paper or ... 
%\geometry{landscape}                		% Activate for for rotated page geometry
%\usepackage[parfill]{parskip}    		% Activate to begin paragraphs with an empty line rather than an indent
\usepackage{graphicx}				% Use pdf, png, jpg, or eps§ with pdflatex; use eps in DVI mode
% TeX will automatically convert eps --> pdf in pdflatex	
\usepackage{amssymb}
\usepackage{float}
\usepackage[cmex10]{amsmath}
\usepackage{mathrsfs,amsthm}
\usepackage{cleveref}
\usepackage{color}
\def\MFIE{{\mathfrak M}}%
\def\EFIE{{\mathfrak E}}%

%\usepackage[sort&compress]{natbib}

\DeclareMathOperator\erf{erf}
\DeclareMathOperator\Ei{Ei}
\newcommand\bbR{\mathbb R}
\newcommand\bphi{\boldsymbol \phi}
\newcommand\brho{\boldsymbol \rho}
\newcommand\rect{\rm rect}
\newcommand\bs{\boldsymbol s}
\newcommand\bx{\boldsymbol x}
\newcommand\bc{\boldsymbol c}
\newcommand\by{\boldsymbol y}
\newcommand\bE{\boldsymbol E}
\newcommand\ds{\boldsymbol ds}
\newcommand\bH{\boldsymbol H}
\newcommand\bF{\boldsymbol F}
\newcommand\bA{\boldsymbol A}
\newcommand\bJ{\boldsymbol J}
\newcommand\bj{\boldsymbol j}
\newcommand\bb{\boldsymbol b}
\newcommand\ba{\boldsymbol a}
\newcommand\bM{\boldsymbol M}
\newcommand\In{\operatorname{inc}}
\newcommand\Sc{\operatorname{scat}}
\newcommand\bn{\boldsymbol n}
\newcommand\br{\boldsymbol r}
\newcommand\nskel{n_{\textrm{skel}}}
\newcommand\ndis{n_{\textrm{dis}}}
\newcommand\eskel{\varepsilon_{\textrm{skel}}}
\newcommand\esmooth{\varepsilon_{\textrm{smooth}}}

\newcommand\bP{\boldsymbol P}
\newcommand\bN{\boldsymbol N}
\newcommand\bQ{\boldsymbol Q}
\newcommand\bX{\boldsymbol X}
\newcommand\bY{\boldsymbol Y}
\newcommand\bV{\boldsymbol V}
\newcommand\bu{\boldsymbol u}
\newcommand\bv{\boldsymbol v}


%%\newcommand\tot{\operatorname{tot}}
\newcommand\tot{\operatorname{}}
%\newtheorem{theorem}{Theorem}
\newtheorem{remark}{Remark}
\newtheorem{definition}{Definition}
\newtheorem{thm}{Theorem}
\newtheorem{lemma}{Lemma}

\newcommand{\Vo}{\mathcal V_0}
\newcommand{\To}{\mathcal T_0}
\newcommand{\Vw}{\mathcal V_\omega}
\newcommand{\Tw}{\mathcal T_\omega}


\title{Notes on Smooth Surfaces}
%\author{Felipe Vico\thanks{Instituto de Telecomunicaciones y Aplicaciones Multimedia (ITEAM), 
%Universidad Polit\` ecnica
%de Val\` encia, 46022 Val\` encia, Spain. {{\em email}: {\sf {felipe.vico@gmail.com,
%}}} } \and
%Leslie Greengard, Michael O'Neil \thanks{Courant Institute of Mathematical Sciences,
%         New York University, 
%         251 Mercer Street,
%         New York, NY 10012-1110.
%{{\em email}: {\sf {greengard@cims.nyu.edu.}}}}
%}

%\date{}							% Activate to display a given date or no date


\begin{document}
\maketitle
%\tableofcontents
%\section{}
%\subsection{}

\section{Three options for $\sigma(\bx)$}
Let $\sigma_{j} = |T_{j}|/\lambda$, 
where $T_{j}$ is the max of the three side lengths of the skeleton triangle,
and $\lambda$ is a constant. Let $\sigma_{0} = \max_{j} \sigma_{j}$.

\begin{align}
\sigma(\bx) &= \sum_{j=1}^{M} \sigma_{j}/M \, ,\\ 
\sigma(\bx) &= \frac{\sum_{j=1}^{M} \sigma_{j} e^{-\| \bx -\bc_{j} \|^2/(2\sigma_{0}^2)}}
{\sum_{j=1}^{M} e^{-\|\bx-\bc_{j}^2 \|/(2\sigma_{0}^2)}} \, ,\\ 
\sigma(\bx) &= \frac{\sum_{j=1}^{M} \sigma_{j} e^{-\| \bx -\bc_{j} \|^2/(2\sigma(\bx)^2)}}
{\sum_{j=1}^{M} e^{-\|\bx-\bc_{j}^2 \|/(2\sigma(\bx)^2)}} \, = \frac{\tilde{F}(\bx,\sigma(\bx))}{\tilde{D}(\bx,\sigma(\bx))} \, .
\label{eq:adapt-flag2}
\end{align}
The strategies have been labeled adapt-flag = 0, 1, and 2 respectively.

\section{Old code understanding}
For adapt-flag =2, the old code ran the following iteration. Let $\alpha(\bx) = 1/(2\sigma(\bx)^2)$,
and let 
\begin{equation}
F(\bx,\alpha) = \sum_{j=1}^{M} \sigma_{j} e^{-\alpha \| \bx -\bc_{j} \|^2} \, , \quad
D(\bx,\alpha) = \sum_{j=1}^{M}  e^{-\alpha \| \bx -\bc_{j} \|^2} \, . 
\end{equation}
\begin{equation}
\alpha^{n+1}(\bx) = \frac{1}{100}\cdot \frac{D(\bx,\alpha^{n}(\bx))^2}{F(\bx,\alpha^{n}(\bx))^2} \, ,
\end{equation}
where the iteration is stopped when
\begin{equation}
\left|\frac{F(\bx,\alpha^{n}(\bx))}{D(\bx,\alpha^{n}(\bx))} - 
\frac{F(\bx,\alpha^{n+1}(\bx))}{D(\bx,\alpha^{n+1}(\bx))} \right| \leq \varepsilon \, .
\end{equation}

Thus, the iterative procedure is computing the solution to the following equation
\begin{equation}
\frac{1}{10\sqrt{\alpha}} = \frac{F(\bx,\alpha(\bx))}{D(\bx,\alpha(\bx))} \, ,
\end{equation}
which in terms of $\sigma(\bx)$ is the iteration
\begin{equation}
\frac{\sqrt{2}\sigma(\bx)}{10} = \frac{\tilde{F}(\bx,\sigma(\bx))}{\tilde{D}(\bx,\sigma(\bx))} \, .
\end{equation}
This is equivalent to rescaling $\lambda \to \sqrt{2}\lambda /10$. 
In Felipe's code, he sets $\lambda = 10$, which in terms of~\cref{eq:adapt-flag2}
corresponds to $\lambda = \sqrt{2}$. 

\section{New code}
The new code runs a damped secant iteration for each $\bx$ to determine 
$\sigma(\bx)$.
Let the iterates for $\sigma(\bx)$ be denoted by
$\sigma_{0}, \sigma_{1}, \ldots \sigma_{n} \ldots$.
Some more simplifying notation. 
Let $\tilde{F}(\bx,\sigma(\bx))/\tilde{D}(\bx,\sigma(\bx)) = H(\bx,\sigma(\bx))$.

Recall that we wish to compute the solution $\sigma(\bx)$ to
$H(\bx,\sigma(\bx)) - \sigma(\bx) = 0$
The two initial guesses for the secant iteration are generated as follows.
Let $s = \max_{j} \sigma_{j}$, then
$\sigma_{0} = H(\bx,s)$, and
$\sigma_{1} = H(\bx,s/2)$.

Then we run the following secant iteration to determine
$\sigma_{n+1}$, given $\sigma_{n}, \sigma_{n-1}$.
\begin{equation}
\sigma_{n+1} = \sigma_{n} + d \, , 
\end{equation}
where $d$ is given by
\begin{equation}
d = -\frac{H(\bx,\sigma_{n}) - \sigma_{n}}{\left(-1 + \frac{H(\bx,\sigma_{n}) - H(\bx,\sigma_{n-1})}{\sigma_{n}-\sigma_{n-1}} \right)}  \, ,
\end{equation}

If $|H(\bx,\sigma_{n+1}) - \sigma_{n+1})| \geq |H(\bx,\sigma_{n}) - \sigma_{n}|$, 
then we set 
$\sigma_{n+1} = \sigma_{n} + d/2$.
We repeat until $\sigma_{n+1} = \sigma_{n} + d/8$, and move to the next
iterate whenever
$|H(\bx,\sigma_{n+1}) - \sigma_{n+1}| < |H(\bx,\sigma_{n}) - \sigma_{n}|$.
If the condition is not satisfied by adding d/8, we declare that the secant
iteration did not converge.
We run the secant iteration until $|H(\bx,\sigma_{n}) - \sigma_{n}| \leq 10^{-14}$,
or a maximum of 100 iterations.

\begin{remark}
The new code breaks for adapt flag=1, currently under investigation
\end{remark}

\section{Results- Gauss identity}
We now compute $\sigma$ by equation~\cref{eq:adapt-flag2} and
use three different values of $\lambda=2.5,5,10$ and compute
the flux due to a point charge on the smooth surface and the skeleton
corresponding to a sphere discretized with 416 second order triangles ($q=2$) 
or 834 first order triangles ($q=1$). af denotes adapt flag.

The function $F$ whose level set describes the smooth surface is 
described by a quadrature on the skeleton 
using $\nskel$ order Vioreanu Rokhlin nodes to compute the level-set function,
and the smooth-surface is described by a collection of $M$ patches
where each patch is discretized using $\ndis$ order Vioreanu Rokhlin nodes.
For fixed $\nskel$, we expect spectral convergence in $\ndis$.
\begin{table}[!ht]
\begin{center}
\begin{tabular}{|c|c|c|c|c|c|c|}
\hline
$\nskel$ & $\ndis$ & $\lambda$ & af & q & $\eskel$ & $\esmooth$ \\ \hline
16 & 16 & 2.5 & 0 & 2 & 7.1e-15 & 6.4e-15 \\ \hline 
16 & 16 & 5.0 & 0 & 2 & 7.1e-15 & 6.0e-15 \\ \hline 
16 & 16 & 10 & 0 & 2 & 7.1e-15 & 2.8e-10 \\ \hline 
16 & 16 & 2.5 & 0 & 1 & 2.7e-15 & 1.3e-15 \\ \hline 
16 & 16 & 5.0 & 0 & 1 & 2.7e-15 & 6.1e-14 \\ \hline 
16 & 16 & 10 & 0 & 1 & 2.7e-15 & 8.5e-12 \\ \hline 
16 & 16 & 2.5 & 1 & 2 & 7.1e-15 & 4.4e-16 \\ \hline 
16 & 16 & 5.0 & 1 & 2 & 7.1e-15 & 2.4e-13 \\ \hline 
16 & 16 & 10 & 1 & 2 & 7.1e-15 & 2.6e-09 \\ \hline 
16 & 16 & 2.5 & 1 & 1 & 2.7e-15 & 1.6e-15 \\ \hline 
16 & 16 & 5.0 & 1 & 1 & 2.7e-15 & 1.7e-12 \\ \hline 
16 & 16 & 10 & 1 & 1 & 2.7e-15 & 9.3e-09 \\ \hline 

\end{tabular}
\caption{Varying adapt-flag}
\end{center}
\end{table}

\begin{table}[!ht]
\begin{minipage}{.3\linewidth}
\caption{$\lambda = 2.5$}
\centering
{\tiny
\begin{tabular}{|c|c|c|c|}
\hline
$\nskel$ & $\ndis$ & $\eskel$ & $\esmooth$ \\ \hline
4 & 4 & 5.1e-10 & 4.5e-08 \\ \hline 
4 & 8 & 5.1e-10 & 5.5e-11 \\ \hline 
4 & 12 & 5.1e-10 & 7.8e-14 \\ \hline 
4 & 16 & 5.1e-10 & 4.4e-16 \\ \hline 
4 & 20 & 5.1e-10 & 6.7e-16 \\ \hline 
8 & 4 & 1.3e-15 & 4.5e-08 \\ \hline 
8 & 8 & 1.3e-15 & 5.5e-11 \\ \hline 
8 & 12 & 1.3e-15 & 7.6e-14 \\ \hline 
8 & 16 & 1.3e-15 & 1.1e-14 \\ \hline 
8 & 20 & 1.3e-15 & 1.3e-14 \\ \hline 
12 & 4 & 8.9e-16 & 4.5e-08 \\ \hline 
12 & 8 & 8.9e-16 & 5.5e-11 \\ \hline 
12 & 12 & 8.9e-16 & 6.8e-14 \\ \hline 
12 & 16 & 8.9e-16 & 4.7e-15 \\ \hline 
12 & 20 & 8.9e-16 & 8.3e-15 \\ \hline 
16 & 4 & 7.1e-15 & 4.5e-08 \\ \hline 
16 & 8 & 7.1e-15 & 5.5e-11 \\ \hline 
16 & 12 & 7.1e-15 & 6.8e-14 \\ \hline 
16 & 16 & 7.1e-15 & 4.2e-15 \\ \hline 
16 & 20 & 7.1e-15 & 8.7e-15 \\ \hline 

\end{tabular}
}
\end{minipage} \hspace{3ex} 
\begin{minipage}{.3\linewidth}
\caption{$\lambda = 5$}
\centering
{\tiny
\begin{tabular}{|c|c|c|c|}
\hline
$\nskel$ & $\ndis$ & $\eskel$ & $\esmooth$ \\ \hline
4 & 4 & 5.1e-10 & 5.7e-07 \\ \hline 
4 & 8 & 5.1e-10 & 1.3e-08 \\ \hline 
4 & 12 & 5.1e-10 & 3.4e-10 \\ \hline 
4 & 16 & 5.1e-10 & 1.1e-09 \\ \hline 
4 & 20 & 5.1e-10 & 6.8e-10 \\ \hline 
8 & 4 & 1.3e-15 & 5.6e-07 \\ \hline 
8 & 8 & 1.3e-15 & 1.3e-08 \\ \hline 
8 & 12 & 1.3e-15 & 3.8e-10 \\ \hline 
8 & 16 & 1.3e-15 & 1.0e-09 \\ \hline 
8 & 20 & 1.3e-15 & 6.9e-10 \\ \hline 
12 & 4 & 8.9e-16 & 5.6e-07 \\ \hline 
12 & 8 & 8.9e-16 & 1.3e-08 \\ \hline 
12 & 12 & 8.9e-16 & 3.8e-10 \\ \hline 
12 & 16 & 8.9e-16 & 1.0e-09 \\ \hline 
12 & 20 & 8.9e-16 & 6.9e-10 \\ \hline 
16 & 4 & 7.1e-15 & 5.6e-07 \\ \hline 
16 & 8 & 7.1e-15 & 1.3e-08 \\ \hline 
16 & 12 & 7.1e-15 & 3.8e-10 \\ \hline 
16 & 16 & 7.1e-15 & 1.0e-09 \\ \hline 
16 & 20 & 7.1e-15 & 6.9e-10 \\ \hline 
20 & 4 & 1.3e-15 & 5.6e-07 \\ \hline 
20 & 8 & 1.3e-15 & 1.3e-08 \\ \hline 
20 & 12 & 1.3e-15 & 3.8e-10 \\ \hline 
20 & 16 & 1.3e-15 & 1.0e-09 \\ \hline 
20 & 20 & 1.3e-15 & 6.9e-10 \\ \hline 

\end{tabular}
}
\end{minipage} \hspace{3ex}
\begin{minipage}{.3\linewidth}
\caption{$\lambda = 10$}
\centering
\centering
{\tiny
\begin{tabular}{|c|c|c|c|}
\hline
$\nskel$ & $\ndis$ & $\eskel$ & $\esmooth$ \\ \hline
4 & 4 & 5.1e-10 & 3.9e-07 \\ \hline 
4 & 8 & 5.1e-10 & 1.4e-07 \\ \hline 
4 & 12 & 5.1e-10 & 1.4e-08 \\ \hline 
4 & 16 & 5.1e-10 & 1.0e-07 \\ \hline 
4 & 20 & 5.1e-10 & 3.4e-08 \\ \hline 
8 & 4 & 1.3e-15 & 1.9e-07 \\ \hline 
8 & 8 & 1.3e-15 & 2.2e-07 \\ \hline 
8 & 12 & 1.3e-15 & 3.2e-08 \\ \hline 
8 & 16 & 1.3e-15 & 9.4e-08 \\ \hline 
8 & 20 & 1.3e-15 & 3.8e-08 \\ \hline 
12 & 4 & 8.9e-16 & 1.8e-07 \\ \hline 
12 & 8 & 8.9e-16 & 2.1e-07 \\ \hline 
12 & 12 & 8.9e-16 & 3.1e-08 \\ \hline 
12 & 16 & 8.9e-16 & 9.4e-08 \\ \hline 
12 & 20 & 8.9e-16 & 4.0e-08 \\ \hline 
16 & 4 & 7.1e-15 & 1.8e-07 \\ \hline 
16 & 8 & 7.1e-15 & 2.1e-07 \\ \hline 
16 & 12 & 7.1e-15 & 3.2e-08 \\ \hline 
16 & 16 & 7.1e-15 & 9.4e-08 \\ \hline 
16 & 20 & 7.1e-15 & 4.0e-08 \\ \hline 
20 & 4 & 1.3e-15 & 1.8e-07 \\ \hline 
20 & 8 & 1.3e-15 & 2.1e-07 \\ \hline 
20 & 12 & 1.3e-15 & 3.2e-08 \\ \hline 
20 & 16 & 1.3e-15 & 9.4e-08 \\ \hline 
20 & 20 & 1.3e-15 & 4.0e-08 \\ \hline 

\end{tabular}
}
\end{minipage}
\caption{Varying $\nskel$ and $\ndis$ for $q=2$, af=2.}
\end{table}

\end{document}
